\pagenumbering{roman}
\section*{Introduction}
Ce  dictionnaire décrit le lexique de la langue japhug (\ipa{kɯrɯ skɤt}), parlée dans la région de Japhug (\zh{茶堡}, \ipa{tɕɤpʰɯ}) au district de Mbarkhams (\zh{马尔康县}), préfecture de Rngaba (\zh{阿坝州}) au Sichuan en Chine, dans les cantons de Gdongbrgyad (\zh{龙尔甲乡}, \ipa{ʁdɯrɟɤt}), Gsarrdzong (\zh{沙尔宗乡},   \ipa{sarndzu}) de Datshang (\zh{大藏乡}, \ipa{tatshi}). Seul le dialecte de Kamnyu (\zh{干木鸟村}, \ipa{kɤmɲɯ}) est représenté dans ce dictionnaire. Cette langue a déjà fait l'objet d'une courte description grammaticale (\citealt{jacques08}) ainsi que d'un recueil d'histoires traditionnelles (\citealt{jacques10gesar}). Un corpus de textes plus important est en cours de publication sur l'archive Pangloss (\citealt{michailovsky14pangloss}).

Ce travail est basé sur les matériaux recueillis à Mbarkhams par l'auteur auprès de Tshendzin (Chenzhen \zh{陈珍}) et Dpalcan (Baierqing \zh{柏尔青}) depuis juillet 2002. Une grande partie des mots, en particulier les verbes et les idéophones, sont illustrés par des exemples enregistrés représentatifs, dont certains proviennent de conversations ou d'histoires traditionnelles.

Chaque entrée du dictionnaire contient une définition en français et en chinois ainsi que la partie du discours du mot, parmi les suivantes:

\begin{itemize}
\item \textit{adv} adverbe
\item \textit{clf} classificateur
\item \textit{idph} idéophone
\item \textit{intj} interjection
\item \textit{n} nom
\item \textit{np} nom inaliénablement possédé 
\item \textit{postp} postposition
\item \textit{pro} pronom
\item \textit{vi} verbe intransitif
\item \textit{vinh} verbe intransitif sans sujet humain
\item \textit{vi-t} verbe semi-transitif
\item \textit{vs} verbe statif
\item \textit{vt} verbe transitif
\item \textit{pc(x,y)} prédicat complexe 
\end{itemize}

Les parties du discours des premiers et deuxièmes éléments des prédicats complexes sont respectivement \ipa{x} et \ipa{y}. Par exemple \ipa{loʁ,tɯ-ʑi} \textit{pc(vs,np)} `avoir la nausée' signifie que l'élément \ipa{loʁ} est morphologiquement un verbe statif, et \ipa{tɯ-ʑi} un nom possédé.

Le numéro qui suit \textit{idph} correspond au patron idéophonique (selon la classification décrite dans \citealt{japhug14ideophones}).

Le verbes contiennent après \textit{dir} le ou les préfixes directionnels utilisés pour former les tiroirs verbaux (décrits dans \citealt[267-9]{jacques14linking}). Le symbole \_  est utilisé pour les verbes de mouvement, de manipulation ou d'action concrète compatibles avec les sept séries de préfixes. Pour les verbes irréguliers (tels que \ipa{ɕe} `aller' ou \ipa{ɣɤʑu}  `exister'), les formes non-prévisibles (thème du passé, seconde personne ou générique). 

Les dérivations verbales sont indiquées par les abréviations suivantes (voir \citealt{jacques12incorp, jacques13tropative, jacques14antipassive,  jacques15spontaneous, jacques15causative}):

\begin{itemize}
\item \textsc{acaus} anticausatif 
\item \textsc{apass} antipassif
\item \textsc{appl} applicatif
\item \textsc{autoben} autobénéfactif-spontané
\item \textsc{caus} causatif 
\item \textsc{comp} composé
\item \textsc{deexp} dé-expérienceur
\item \textsc{deidph} déidéophonique
\item \textsc{denom} dénominal
\item \textsc{facil} facilitatif
\item \textsc{incorp} incorporation
\item \textsc{n.orient} action non-orientée
\item \textsc{pass} passif
\item \textsc{recip} réciproque
\item \textsc{refl} réfléchi
\item \textsc{trop} tropatif
\item \textsc{vert} vertitif
\end{itemize}

Ce dictionnaire a bénéficié des corrections de nombreux collègues et étudiants, en particulier Gong Xun, Peng Guozhen, Zhang Shuya.

Ce travail a été financé par le projet ANR HimalCo  (ANR-12-CORP-0006) et est en relation avec le projet de recherche LR-4.11 ‘‘Automatic Paradigm Generation and Language Description’’ du Labex EFL (fondé par l'ANR/CGI)《

\newpage
\begin{thebibliography}{10}

\providecommand{\natexlab}[1]{#1}
\providecommand{\url}[1]{#1}
\providecommand{\urlprefix}{}
\expandafter\ifx\csname urlstyle\endcsname\relax
\providecommand{\doi}[1]{doi:\discretionary{}{}{}#1}\else
\providecommand{\doi}{doi:\discretionary{}{}{}\begingroup
\urlstyle{rm}\Url}\fi

\bibitem[{Jacques(2008)}]{jacques08}
Jacques, Guillaume. 2008.
\newblock \emph{{J}iarongyu yanjiu \zh{嘉絨語研究} ({S}tudy on the
  {R}gyalrong language)}.
\newblock Beijing: Minzu chubanshe.

\bibitem[{Jacques(2012)}]{jacques12incorp}
Jacques, Guillaume. 2012.
\newblock {F}rom denominal derivation to incorporation.
\newblock \emph{Lingua} 122.11. 1207--1231.

\bibitem[{Jacques(2013{\natexlab{a}})}]{jacques13tropative}
Jacques, Guillaume. 2013{\natexlab{a}}.
\newblock {A}pplicative and tropative derivations in {J}aphug {R}gyalrong.
\newblock \emph{Linguistics of the Tibeto-Burman Area} 36.2. 1--13.

\bibitem[{Jacques(2013{\natexlab{b}})}]{japhug14ideophones}
Jacques, Guillaume. 2013{\natexlab{b}}.
\newblock {I}deophones in {J}aphug {R}gyalrong.
\newblock \emph{Anthropological Linguistics} 55.3. 256--287.

\bibitem[{Jacques(2014{\natexlab{a}})}]{jacques14linking}
Jacques, Guillaume. 2014{\natexlab{a}}.
\newblock {C}lause linking in {J}aphug {R}gyalrong.
\newblock \emph{Linguistics of the Tibeto-Burman Area} 37.2. 263--327.

\bibitem[{Jacques(2014{\natexlab{b}})}]{jacques14antipassive}
Jacques, Guillaume. 2014{\natexlab{b}}.
\newblock {D}enominal affixes as sources of antipassive markers in {J}aphug
  {R}gyalrong.
\newblock \emph{Lingua} 138. 1--22.

\bibitem[{Jacques(to appear{(\natexlab{a})})}]{jacques15causative}
Jacques, Guillaume. to appear{(\natexlab{a})}.
\newblock {T}he origin of the causative prefix in {R}gyalrong languages and its
  implication for proto-{S}ino-{T}ibetan reconstruction.
\newblock \emph{Folia Linguistica Historica} .

\bibitem[{Jacques(to appear{(\natexlab{b})})}]{jacques15spontaneous}
Jacques, Guillaume. to appear{(\natexlab{b})}.
\newblock {T}he spontaneous- autobenefactive prefix in {J}aphug {R}gyalrong.
\newblock \emph{Linguistics of the Tibeto Burman Area} .

\bibitem[{Jacques \& Chen(2010)}]{jacques10gesar}
Jacques, Guillaume \& Zhen Chen. 2010.
\newblock \emph{{U}ne version rgyalrong de l'épopée de {G}esar}.
\newblock Osaka: National Museum of Ethnology.

\bibitem[{Michailovsky et~al.(2014)Michailovsky, Mazaudon, Michaud, Guillaume,
  François \& Adamou}]{michailovsky14pangloss}
Michailovsky, Boyd, Martine Mazaudon, Alexis Michaud, Séverine Guillaume,
  Alexandre François \& Evangelia Adamou. 2014.
\newblock {D}ocumenting and researching endangered languages: the {P}angloss
  {C}ollection.
\newblock \emph{Language Documentation and Conservation} 8. 119–135.

\end{thebibliography}
\cleardoublepage
\pagenumbering{arabic}
\setmainfont[Mapping=tex-text,Numbers=OldStyle,Ligatures=Common]{Charis SIL} 